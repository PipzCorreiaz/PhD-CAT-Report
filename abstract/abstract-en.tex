%!TEX root = ../dissertation.tex

\begin{otherlanguage}{english}
\begin{abstract}
% Set the page style to show the page number
\thispagestyle{plain}
\abstractEnglishPageNumber
Recently, the field of Human-Robot Interaction (HRI) has been paying close attention to group interactions. They refer to multi-party settings that extend (traditional) dyads of one person and one robot, to cases where there are multiple people and/or multiple robots. This PhD thesis will focus on the challenges of creating social robots that are capable of sustaining cohesive alliances in team settings with humans. It explores different dimensions of group cohesion, namely: social cohesion, collective cohesion, and structural cohesion. The contributions include empirical user studies analysing membership preferences (i.e., social cohesion) and group identification (i.e., collective cohesion) and how those can be influenced by different social behaviours of robotic teammates or by other factor e.g., the outcome of the team. Moreover, it contributes with computational mechanisms for the robotic teammate autonomously express group-based emotions. Finally, the proposed work aims to investigate structural cohesion and computational mechanisms to improve the perceptive capabilities of an autonomous robotic teammate.
% Keywords
\begin{flushleft}

\keywords{}

\end{flushleft}

\end{abstract}
\end{otherlanguage}