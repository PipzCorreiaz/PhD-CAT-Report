%!TEX root = ../dissertation.tex

\chapter{Conclusion and Plan of Work}
\label{chapter:conclusion}

The current document reports an on-going dissertation on the topic of robotic teammates. In particular, this thesis explores the following research problem:

%reinforce why cohesion

\begin{indented}
How can we endow a social robot with the ability to improve the cohesive alliance in a team setting with humans?
\end{indented}

Our approach to address this problem consists of exploring two research goals. One of them is focused on investigating how human-robot teams are established, in particular analysing the perspective of human teammates on how should social robots integrate these teams. It involves the understanding of how people perceive robotic teammates, what people expect from them, or even what attitudes and predispositions people have towards them. Additionally, this research goal raises interesting challenges related to methodological aspects of analysing group interactions. From the careful selection of the group measures, to the adequate design of experiments, or even the correct statistical analysis of individual- and group-level factors.

The second research goal, on the other hand, is focused on the perspective of the social robot and what computational mechanisms it can apply in order to increase the effectiveness of the team. Such goal may also involve several challenges from the autonomous decision-making process, to its perceptive capabilities, or even to the authoring of its social behaviours. The current contributions to this research goal explore novel approaches for the decision-making of robotic teammates.

Overall, 

Finally, we would like to highlight the following publications that are explicitly linked to the goals of this dissertation:
\begin{itemize}
    \item Correia, F., Petisca, S., Alves-Oliveira, P., Ribeiro, T., Melo, F. S., \& Paiva, A. (2017, July). Groups of humans and robots: Understanding membership preferences and team formation. In Robotics: Science and Systems (RSS).
    \item Correia, F., Petisca, S., Alves-Oliveira, P., Ribeiro, T., Melo, F. S., \& Paiva, A. (2019). ``I Choose... YOU!'' Membership preferences in human–robot teams. Autonomous Robots, 43(2), 359-373.
    \item Correia, F., Mascarenhas, S. F., Gomes, S., Arriaga, P., Leite, I., Prada, R., Melo, F. S. \& Paiva, A. (2019, March). Exploring prosociality in human-robot teams. In 2019 14th ACM/IEEE International Conference on Human-Robot Interaction (HRI) (pp. 143-151). IEEE.
    \item Correia, F., Mascarenhas, S., Prada, R., Melo, F. S., \& Paiva, A. (2018, February). Group-based emotions in teams of humans and robots. In Proceedings of the 2018 ACM/IEEE International Conference on Human-Robot Interaction (pp. 261-269). ACM.
    \item Correia, F., Melo, F. S., \& Paiva, A. (2019, March). Group Intelligence on Social Robots. In 2019 14th ACM/IEEE International Conference on Human-Robot Interaction (HRI Pioneers Workshop) (pp. 703-705). IEEE.
\end{itemize}


Furthermore, beyond the focus of this dissertation on establishing cohesive alliances with robotic teammates, we also contributed to the HRI field with additional publications. The following list contains only the publications involving group interactions, which are related to the scope of this dissertation:
\begin{itemize}
    \item Oliveira, R., Arriaga, P., Alves-Oliveira, P., Correia, F., Petisca, S., \& Paiva, A. (2018, February). Friends or foes?: Socioemotional support and gaze behaviors in mixed groups of humans and robots. In Proceedings of the 2018 ACM/IEEE International Conference on Human-Robot Interaction (pp. 279-288). ACM.
    \item Oliveira, R., Arriaga, P., Correia, F. \& Paiva. A. (2019, March) Looking Beyond Collaboration: Socioemotional Positive, Negative and Task-oriented Behaviors in Human-Robot Group Interactions. In International Journal of Social Robotics
    \item Oliveira, R., Arriaga, P., Correia, F., \& Paiva, A. (2019, March). The Stereotype Content Model Applied to Human-Robot Interactions in Groups. In 2019 14th ACM/IEEE International Conference on Human-Robot Interaction (HRI) (pp. 123-132). IEEE.
    \item Correia, F., Chandra, S., Mascarenhas, S., Charles-Nicolas, J., Gally, J., Lopes, D., Santos, F. P., Santos, F. C., Melo, F. S. \& Paiva, A. (2019, October). Walk the Talk! Exploring (Mis)Alignment of Words and Deeds by Robotic Teammates in a Public Goods Game. In 28th IEEE international symposium on robot and human interactive communication (RO-MAN). IEEE.
    \item Correia, F., Mascarenhas, S., Gomes, S., Melo, F. S., \& Paiva, A. The Dark Side of Embodiment - Teaming Up With Robots VS Disembodied Agents. (2020, SUBMITTED).
\end{itemize}


\section{Work Plan}
%Additionally, the future work we have proposed in Chapter~\ref{chapter:future-work} also aims at contributing to this research goal by enhancing the perceptive capabilities of robotic teammates.
According to the proposed work detailed in Chapter~\ref{chapter:future-work}, we are currently considering a sequence of three user studies. \textit{Study 1} aims at collecting data from human participants playing a collaborative game and negotiating with team members in order to later develop the behaviours of a robotic teammate that will play the same game. In \textit{Study 2}, human participants will play together with a robot and the data collection will allow us to analyse communication patterns of the team when there is a robotic partner. Finally, the \textit{Study 3} will explore adaptive mechanisms on the behaviours of the robot and their impact on the communication network of the team.


\begin{table}[h]
\centering
\caption{Work plan for the proposed work}
\begin{tabular}{|r|l|} 
\hline
\textbf{Winter 2019} & Study 1            \\ 
\hline
\textbf{Spring 2020} & Visit another lab  \\ 
\hline
\textbf{Summer 2020} & Study 2            \\ 
\hline
\textbf{Fall 2020}   & Study 3            \\ 
\hline
\textbf{Winter 2020} & Writing thesis     \\
\hline
\end{tabular}
\end{table}

\section*{\centering*}
This dissertation proposes two research goals that address the research problem by two different perspectives. These research goals hold complementary and, at a certain extent, interdependent challenges that reflect the interdisciplinarity of the HRI field. Overall, we believe its contributions strongly support and enhance research on group interactions with robotic teammates.